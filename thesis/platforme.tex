\chapter{Pregled platform}

Danes lahko grafično intenzivne aplikacije poganjamo na ogromnem naboru različnih naprav. Grafično pospeševanje izrisovanja 3D objektov, ki je bilo še pred kratkim omejeno na namizne računalnike in konzole, je sedaj možno tudi na mobilnih telefonih in grafičnih tablicah. Razvoj aplikacij za mobilne platforme je nekoliko bolj zahteven, saj moramo poleg okrnjenega nabora ukazov in slabše strojne zmogljivosti, paziti tudi na porabo baterije.

V zadnjem letu so postale zanimivi računalniki s čipom na eni sami ploščici (prevedi single-board computer). Tak primer je Rasperry Pi, računalnik, ki je bil razvit s namenom promoviranja učenja osnovnih temeljev računalništva. Tudi majhni računalniki kot je omenjeni Raspberry Pi imajo namenski grafični procesor, ki omogoča izvajanje grafično bogatih aplikacij.

\section{Namizni in prenosni računalniki}

Največ svobode pri razvoju grafično intenzivnih aplikacij je na voljo na namiznih računalnikih. Ti imajo na voljo najboljšo možno strojno oprema in tudi operacijski sistemi so zreli in dovršeni, saj so se razvijali že več kot deset let. Namizni računalniki nimajo tako strogih omejitev z velikostjo, kot mobilne naprave. Prav nič presenetljivo ni torej, da so grafično intenzivne aplikacije na tej platformi najbolj domače. 

Namizni računalniki so bili zelo dolgo edina platforma, ki je bila zmožna poganjati grafično intenzivne aplikacije. S pojavom namenskih grafičnih čipov (3Dfx), je prvič postalo mogoče razvijati aplikacije izrisovanje kompleksnejših grafičnih oblik na zaslonu. S pojavom grafičnih kartic se je zmanjšala obremenitev centralne procesne enote. Vse zahtevne operacije\footnote{operacije s plavajočo vejico...} so se sedaj izvajale na ločeni enoti. 

Na namiznih računalnikih sta se v začetku 90ih let pojavili dve knjižnici za delo s 3D grafiko. OpenGL in DirectX. Obe ponujata programski vmesnik za komunikacijo z grafično procesno enoto, vendar je precejšna razlika med njima.

% http://en.wikipedia.org/wiki/Comparison_of_OpenGL_and_Direct3D



\subsection{OpenGL}

OpenGL je odprt standard, ki ga razvija skupina Khronos. Na voljo je na večini operacijskih sistemih (Windows, Mac OSX in Linux). 

Tekom let so izšle 4 verzije. Podpora za programiranje grafičnega vhoda (shaderji) so postali podprti v verziji 2.0, ko je bila narejena tudi totalna prenova celotnega sistema. Iz te prenove se je nato razvila tudi mobilna inačica OpenGL ES 2.0.  

OpenGL je implementiran v gonilniku za zaslon in vsak proizvajalec grafičnih kartic mora v gonilnike za grafično kartico dodati podporo. Problem s tem principom je, da se vmesnik nekoliko razlikuje med različnimi proizvajalci grafičnih kartic. Tudi programski jezik za pisanje shader programov in njihovi prevajalniki se lahko med seboj razlikujejo.

OpenGL se veliko uporablja za profesionalne aplikacije, kot so programi za grafično modeliranje (Maya) in programi za računalniške simulacije...

\subsection{Direct3D}

Direct3D je bil Microsoftov odgovor na OpenGL. Direct3D je zaprt programski vmesnik, popolnoma v lasti Microsofta. Programski vmesnik je uradno podprt samo na operacijskih sistemih Windows in na Microsoftovi igralni konzoli Xbox. Na drugih platformah je možno Direct3D aplikacije poganjati samo z uporabo posebne virtualizacijske plasti. Na Linuxu to virtualizacijsko plast ponuja orodje Wine, ki pa še vedno ne podpira Direct3D vmesnika v celoti.

Direct3D se ne uporablja toliko v profesionalnih aplikacijah kot OpenGL. Razlog je bil v slabši hitrosti izvajanja na začetku... Je pa toliko bolj popularen pri računalniških igrah.

\section{RaspberryPi}
\label{sec:raspberryPi}

RaspberryPI je majhen računalnik, velikosti kreditne kartice, ki je bil razvit za promocijo učenja računalniške znanosti. Računalnik vsebuje 700 MHz ARM procesor, 256 (ali 512) MB delovnega pomnilnika in grafično procesno enoto VideoCore IV s 250MHz, ki podpira tudi OpenGL ES2.0.

Raspberry Pi je zanimiva mešanica med mobilnimi in namiznimi računalniki. Po svoji strojni opremi je sicer zelo podoben mobilnim napravam, vendar na njem ne teče mobilni operacijski sistem. Na Raspberry Piju je mogoče poganjati operacijske sisteme, ki so značilni za namizne računalnike. Najbolj pogosto uporabljena je malce prirejena verzija Linux distribucije Debian, uradno možno pa je tudi naložiti distribuciji Arch in Fedora.

Čip za izrisovanje grafike na RaspberryPiju ima popolno podporo za OpenGL ES2.0 in je kljub omejitvam zmožen poganjati zahtevnejše aplikacije. Lep primer je računalniška igra Minecraft.

\section{Mobilne platforme} 

Na trgu najdemo pester nabor mobilnih platform. Največji igralci v času pisanja so Google s svojim odprtokodnim sistemom Android, Apple s svojim sistemom iOS, nekaj tržnega deleža pa imata tudi podjetji Microsoft, z mobilno različico opercijskega sistema Windows (Windows 7, 8 RT, Phone), ter podjetje BlackBerry z istoimenskim naborom pametnih telefonov namenjenim predvsem poslovnim uporabnikom.

Poleg že obstoječih pa bodo v bližnji prihodnosti na trg stopili tudi novi igralci. Fundacija Mozilla je razvila Firefox OS, podjetje Canonical pa pripravlja različico Ubuntu operacijskega sistema, ki bo delovala tudi na mobilnih telefonih in tablicah.

\subsection{iOS}

Prvi iPhone je bil predstavljen 9. januarja 2007. Od ostalih mobilnih naprav na tržišču se je razlikoval z dodelanim uporabniškim vmesnikom in zaslonom občutljivim za večprstne dotike. Bil je tudi eden izmed prvih mobilnih telefonov brez tipkovnice in ostalih fizičnih gumbov - uporabnik je do vseh funkcij telefona dostopal preko na dotik občutljivega zaslona in enega fizičnega gumba.

V nasljednih letih je podjetje Apple Inc. dodalo prvemu telefonu nekaj dodatnih funkcij kot so 3g povezljivost, izboljšana zadnja kamera, zaslon z višjo resolucijo, dvo jederni procesor itd.

Naprave danes podpirajo OpenGL ES 1.1 in 2.0, planirana pa je tudi podpora OpenGL ES3.0. 

Razvoj aplikacij poteka v jeziku Objective C (objektni C). Jezik temelji na ANSI Cju, vendar z dodano podporo objektno orientiranim konceptom. Prevajalnik za Objective C lahko prevede vsak program napisan v Cju. Objektno orientirani koncepti so implementirani s konceptom pošiljanja sporočil, slično programskemu jeziku SmallTalk. Za razliko od Jave, ki se uporablja na sistemih Android, Objective C nima avtomatičnega sproščanja pomnilnika, kar pri razvoju zahtevnih aplikacij lahko štejemo kot prednost, saj ima programer na voljo več orodij za delo s pomnilnikom. 

Problem pri razvijanju iOS aplikacij je v tem, da je razvoj možen samo na strojni in programski opremi, ki jo proizvaja Apple. Tako za Android in Windows Phone je možno razvijati na konkurenčnih operacijskih sistemih.

\subsection{Android}

Android je operacijski sistem, ki temelji na Linux jedru. Operacijski sistem je razvilo podjetje Android Inc., s finančno pomočjo Googla, ki je leta 2005 primarno podjetje tudi kupil. Prvi mobilni telefon z Android operacijskim sistemom je bil prodan oktobra 2008.

Izvorna koda operacijskega sistema je odprta in dostopna pod Apache licenco.

Android podpira OpenGL ES 1.1 in 2.0 od verzije 2.2 dalje. Verzija 4.3 pa prinaša podporo tudi za OpenGL ES3.0. Podpora za OpenGL ES 2.0 na verziji 2.2 ni popolna in je potrebno napisati lasten C++ wrapper, ki omogoči funkcionalnosti, ki jo API ne podpira.


% http://en.wikipedia.org/wiki/Dalvik_(software)
Programski jezik za razvoj domorodnih aplikacij na sistemu Android je Java, ki teče na virtualnem stroju Dalvik. Aplikacije napisane v Javi se prevedejo v bitno kodo (bytecode) in se nato iz JVM kompatibilnih .class datotek pretvorijo v .dex datoteke, ki so kompatibilne na Dalviku. Format .dex je namenjen sistemom, ki imajo omejeno količino pomnilnika in procesorske moči.

\subsection{Windows}

% http://msdn.microsoft.com/en-us/library/windowsphone/develop/jj207052(v=vs.105).aspx
% http://msdn.microsoft.com/en-us/library/windowsphone/develop/jj662943(v=vs.105).aspx
% 

Mobilni operacijski sistem Windows 8 Phone (in RT), za dostop do grafične kartice uporabljata Microsoftovo knjižnico Direct3D. Za razliko od drugih mobilnih operacijskih sistemov, je Windows eden izmed redkih, ki ne podpira OpenGL ES programskega vmesnika. Izvajanje OpenGL ES aplikacij je tako mogoče zgolj s posebno virtualizacijo in uporabo orodij, kot je ANGLE.

Programski jezik, ki je uporabljen na Windows mobilnih sistemih je C\#. C\# je bil razvit pri Microsoftu, kot del njihove .NET iniciative. Ecma ga je priznala kot standard 4. julija 2006 (ECMA-334). Podobno kot ObjectiveC je bil tudi C\# razvit z namenom dodajanja objektno orientiranih lastnosti v programski jezik C. Glavni razvijalec programskega jezika C\# je Anders Hejlberg. C\# nekoliko spominja na programski jezik Java, vendar se te dva jezika v kasnejših verzijah kar precej razlikujeta. Lep primer je implementacija genericov, ki je v C\# ustvarjena s pomočjo reifikacijo podatkov, v Javi pa s pomočjo posebne sintakse. 

\subsubsection{ANGLE}
% https://code.google.com/p/angleproject/
ANGLE: Almost Native Graphics Layer Engine
Angle je odprtokodni projekt, ki implementira OpenGL ES2.0 specifikacijo in jo strojno pospeši z Direct3D. Angle je v uporabi kot primarno zaledje za WebGL v brskalnikih Chrome in Firfox. Podpira DirectX9 do DirectX11.  


\subsection{Firefox OS}

Firefox OS je mobilni operacijski sistem, ki temelji na odprtih standardih spleta. Domorodne aplikacije pisane za Firefox OS so kar spletne aplikacije narejene po načelih HTML5, ki imajo posebne ovitke za klicanje sistemskih funkcij, kot je klicanje in dostop do senzorjev za lokacijo, pospeške in tako dalje.

Domorodne aplikacije gradimo z uporabo programskega jezika Javascript, za izgled in obliko aplikacij pa skrbita HTML5 in CSS. Aplikacije lahko z uporabo orodij v SDKju, preizkusimo tudi na mobilnih računalnikih.

Ker podpora temelji na standardih za spletne tehnologije ni podpore za OpenGL ES1.x, ki je na voljo na drugih mobilnih operacijskih sistemih. Je pa seveda podprt OpenGL ES 2.0 v obliki WebGLa.

Operacijski sistem Firefox OS sicer temelji na jedru Linux, ki služi kot platforma na katerem se nato zažene Gecko. Gecko je tako imenovani pogon za razporeditev, ki je prisoten v vseh verzijah brskalnika Firefox in pa tudi drugje. Ker Gecko deluje na različnih platformah, je Firefox OS možno naložiti tudi na druge naprave, tudi na RaspberryPi[\ref{sec:raspberryPi}]. 



\subsection{Ubuntu}

Operacijski sistem Ubuntu je najbolj popularen na Linuxu temelječi operacijski sistem na namiznih računalnikih. Podjetje, ki razvija operacijski sistem Ubuntu ima vizijo spraviti podobno funkcionalnost tudi na mobilne naprave - telefone in tablice. Prva naprava, ki bo izšla z mobilnim operacijskim sistemom Ubuntu, je telefon Edge in je bila napovedana 22. julija 2013. Pred tem so radovedni razvijalci lahko naložili operacijski sistem na mobilni telefon Nexus 4, ki ima privzeto naložen operacijski sistem Android.

% http://developer.ubuntu.com/resources/programming-languages/qml/
Canonical za razvoj aplikacij za mobilni operacijski sistem Ubuntu priporoča uporabo programskega jezika QML (Qt Meta Language). QML je programski jezik namenjen lažjemu razvoju uporabniškega vmesnika. 

Dokumentacija za Ubuntu Phone še ni popolna, ampak kot se da trenutno razbrati je uporaba OpenGLa podprta s strani qt 3d iniciative. Za pisanje grafično intenzivnih aplikacij se tako namesto QMLa priporoča C++.

O samih zmogljivostih telefona in tablic se sicer še ne ve veliko, pričakuje pa se, da bo podprt vsaj standard za OpenGL ES2.0 aplikacije.

\section{Skupne zmogljivosti}

Kot smo videli pri posameznih primerih ima večina mobilnih naprav podporo za OpenGL ES2.0, tako da je glavni problem pri razvoju medplatformnih aplikacijah premagati omejitve pri domorodnih jezikih. Večina mobilnih operacijskih sistemih ima svoj programski jezik, ki ga preferira za pisanje aplikacij.

Večina izmed njih ima tudi na voljo tako imenovani NDK native development kit oz. po slovensko domorodni nabor orodij za razvijanje aplikacij. NDK normalno podpira jezik C++, ki je na voljo na večini operacijskih sistemov. Tudi 




