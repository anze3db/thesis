\chapter{Pregled platform}

Danes lahko grafično intenzivne aplikacije poganjamo na ogromnem naboru različnih naprav. Grafično pospeševanje izrisovanja 3D objektov, ki je bilo še pred kratkim omejeno na namizne računalnike in še pred tem na drage delovne postaje, je sedaj možno tudi na mobilnih telefonih in grafičnih tablicah. Razvoj aplikacij za mobilne platforme je nekoliko bolj zahteven, saj moramo poleg okrnjenega nabora ukazov in slabše strojne zmogljivosti, paziti tudi na porabo baterije.

Na baterijo moramo paziti tudi pri prenosnih računalnikih. Ti so po zmogljivostim podobni namiznim računalnikom in uporabljajo enako (ali vsaj podobno) arhitekturo za centralno procesno enoto.

\section{Namizni in prenosni računalniki}

Največ svobode pri razvoju grafično intenzivnih aplikacij je na voljo na namiznih računalnikih. Ti imajo na voljo najboljšo možno strojno opremo in tudi operacijski sistemi so zreli in dovršeni, saj so v razvoju že več deset let. Namizni računalniki nimajo tako strogih omejitev z velikostjo, kot mobilne naprave. Prav nič presenetljivo ni torej, da so grafično intenzivne aplikacije na tej platformi najbolj domače. 

Namizni računalniki uporabljajo drugačno arhitekturo za centralno procesno enoto. Procesorji v namiznih računalnikih so zgrajeni s CISC (Intel, AMD), medtem ko mobilne večinoma uporabljajo RISC arhitekturo in procesorje podjetja ARM in nVidia. 

Namizni računalniki so bili zelo dolgo edina platforma, na kateri je bilo možno poganjati grafično intenzivne aplikacije. S pojavom namenskih grafičnih čipov (grafične kartice na začetku), je prvič postalo mogoče razvijati aplikacije izrisovanje kompleksnejših grafičnih oblik na zaslonu. S pojavom grafičnih kartic se je zmanjšala obremenitev centralne procesne enote. Vse zahtevne operacije\footnote{operacije s plavajočo vejico...} se danes izvajajo na grafičnih procesnih enotah, ki so narejene z visoko stopnjo paralelizma. 

Na namiznih računalnikih sta se v začetku 90ih let pojavili dve knjižnici za delo s 3D grafiko. OpenGL in Direct3D. Obe ponujata programski vmesnik za komunikacijo z grafično procesno enoto, vendar so razlike med njima precejšne.

% http://en.wikipedia.org/wiki/Comparison_of_OpenGL_and_Direct3D



\subsection{OpenGL}

OpenGL je odprt standard, ki ga razvija skupina Khronos. Na voljo je na večini operacijskih sistemih (Windows, Mac OSX in Linux).

Programski vmesnik se je na začetku uporabljal predvsem za profesionalne aplikacije, kot je na primer AutoCAD in simulacije. Šele kasneje se je programski vmesnik razvil, do te mere, da je bil uporaben tudi za druge namene.

Tekom let so izšle 4 verzije. Podpora za programiranje grafičnega vhoda (shaderji) so postali podprti v verziji 3.0, ko je bila narejena tudi totalna prenova celotnega sistema. Iz te prenove se je nato razvila tudi mobilna inačica OpenGL ES 2.0.  

OpenGL je implementiran v gonilniku za zaslon in vsak proizvajalec grafičnih kartic mora v gonilnike za grafično kartico dodati podporo. Problem s tem principom je, da se vmesnik nekoliko razlikuje med različnimi proizvajalci grafičnih kartic. Tudi programski jezik za pisanje shader programov in njihovi prevajalniki se lahko med seboj razlikujejo.

Dolgo časa so bile tudi razširitve, ki so bile na voljo samo na določenih grafičnih karticah. Proizvajalci grafičnih kartic so na ta način želeli izkazati superiornost, saj so te pritikline navadno bile kot dodaten sladkor pri prikazovanju. To je ustvarilo resen problem, kjer so razvijalci morali prilagajati kodo aplikacij glede na posamezne grafične kartice.


\subsection{Direct3D}

Direct3D je bil Microsoftov odgovor na OpenGL. Direct3D je zaprt programski vmesnik, popolnoma v lasti Microsofta. Programski vmesnik je uradno podprt samo na operacijskih sistemih Windows. Nekoliko spremenjena oblika vmesnika se nahaja tudi na Microsoftovi igralni konzoli Xbox. Na drugih platformah je možno Direct3D aplikacije poganjati samo z uporabo posebne virtualizacijske plasti. Na Linuxu to virtualizacijsko plast ponuja orodje Wine, vendar ne podpira Direct3D vmesnika v celoti.

Direct3D se ne uporablja v profesionalnih aplikacijah tako pogosto kot OpenGL. Razlogov je več. Na začetku je bil OpenGL vmesnik hitrejši in bolj natančen pri izrisovanju. Ker se je OpenGL pojavil pred Direct3Djem se je že zakoreniril. Direct3D je bil za razliko od OpenGL namenjen za osebne računalnike in ne samo za drage delovne postaje. Zaradi fiksnega cevovoda za izrisovanje je onemogočil proizvajalcem grafičnih kartic ustvarjanje lastnih modulov, ki bi otežili razvijanje aplikacij. Programski vmesnih je zaradi teh razlogov postal zelo popularen pri razvijalcih računalniških iger in tudi večina aplikacij, ki API uporablja je računalniška igra.

\section{Mobilne platforme} 

Na trgu najdemo pester nabor mobilnih platform. Največji igralci v času pisanja so Google s svojim odprtokodnim sistemom Android, Apple s svojim sistemom iOS, nekaj tržnega deleža pa imata tudi podjetji Microsoft, z mobilno različico opercijskega sistema Windows (Windows 7, 8 RT, Phone), ter podjetje BlackBerry z istoimenskim naborom pametnih telefonov namenjenim predvsem poslovnim uporabnikom.

Poleg že obstoječih pa bodo v bližnji prihodnosti na trg stopili tudi novi igralci. Fundacija Mozilla je  razvila svojo rešitev - Firefox OS, temelječ na spletnih tehnologijah. Podjetje Canonical pripravlja različico Ubuntu operacijskega sistema, ki bo delovala tudi na mobilnih telefonih in tablicah. Na Finskem pa podjetje Jolla Mobile razvija svoj lastni operacijski sistem Sailfish OS, ki tudi temelji na Linux jedru.

\subsection{iOS}

Prvi iPhone je bil predstavljen 9. januarja 2007. Od ostalih mobilnih naprav na tržišču se je razlikoval z dodelanim uporabniškim vmesnikom in zaslonom občutljivim za večprstne dotike. Bil je tudi eden izmed prvih mobilnih telefonov brez tipkovnice in ostalih fizičnih gumbov - uporabnik je do vseh funkcij telefona dostopal preko na dotik občutljivega zaslona in enega fizičnega gumba.

V nasljednih letih je podjetje Apple Inc. dodalo prvemu telefonu nekaj dodatnih funkcij kot so 3g povezljivost, izboljšana zadnja kamera, zaslon z višjo resolucijo, dvo jederni procesor itd.

Naprave danes podpirajo OpenGL ES 1.1 in 2.0, planirana pa je tudi podpora OpenGL ES3.0. 

Razvoj aplikacij poteka v jeziku Objective C (objektni C). Jezik temelji na ANSI Cju, vendar z dodano podporo objektno orientiranim konceptom. Prevajalnik za Objective C lahko prevede vsak program napisan v Cju. Objektno orientirani koncepti so implementirani s konceptom pošiljanja sporočil, slično programskemu jeziku SmallTalk. Za razliko od Jave, ki se uporablja na sistemih Android, Objective C nima avtomatičnega sproščanja pomnilnika, kar pri razvoju zahtevnih aplikacij lahko štejemo kot prednost, saj ima programer na voljo več orodij za delo s pomnilnikom. 

Problem pri razvijanju iOS aplikacij je v tem, da je razvoj možen samo na strojni in programski opremi, ki jo proizvaja Apple. Tako za Android in Windows Phone je možno razvijati na konkurenčnih operacijskih sistemih.

\subsection{Android}

Android je operacijski sistem, ki temelji na Linux jedru. Operacijski sistem je razvilo podjetje Android Inc., s finančno pomočjo Googla, ki je leta 2005 primarno podjetje tudi kupil. Prvi mobilni telefon z Android operacijskim sistemom je bil prodan oktobra 2008.

Izvorna koda operacijskega sistema je odprta in dostopna pod Apache licenco.

Android podpira OpenGL ES 1.1 in 2.0 od verzije 2.2 dalje. Verzija 4.3 pa prinaša podporo tudi za OpenGL ES3.0. Podpora za OpenGL ES 2.0 na verziji 2.2 ni popolna in je potrebno napisati lasten C++ wrapper, ki omogoči funkcionalnosti, ki jo API ne podpira.


% http://en.wikipedia.org/wiki/Dalvik_(software)
Programski jezik za razvoj domorodnih aplikacij na sistemu Android je Java, ki teče na virtualnem stroju Dalvik. Aplikacije napisane v Javi se prevedejo v bitno kodo (bytecode) in se nato iz JVM kompatibilnih .class datotek pretvorijo v .dex datoteke, ki so kompatibilne na Dalviku. Format .dex je namenjen sistemom, ki imajo omejeno količino pomnilnika in procesorske moči.

\subsection{Windows}

% http://msdn.microsoft.com/en-us/library/windowsphone/develop/jj207052(v=vs.105).aspx
% http://msdn.microsoft.com/en-us/library/windowsphone/develop/jj662943(v=vs.105).aspx
% 

Mobilni operacijski sistem Windows 8 Phone (in RT), za dostop do grafične kartice uporabljata Microsoftovo knjižnico Direct3D. Za razliko od drugih mobilnih operacijskih sistemov, je Windows eden izmed redkih, ki ne podpira OpenGL ES programskega vmesnika. Izvajanje OpenGL ES aplikacij je tako mogoče zgolj s posebno virtualizacijo in uporabo orodij, kot je ANGLE.

Programski jezik, ki je uporabljen na Windows mobilnih sistemih je C\#. C\# je bil razvit pri Microsoftu, kot del njihove .NET iniciative. Ecma ga je priznala kot standard 4. julija 2006 (ECMA-334). Podobno kot ObjectiveC je bil tudi C\# razvit z namenom dodajanja objektno orientiranih lastnosti v programski jezik C. Glavni razvijalec programskega jezika C\# je Anders Hejlberg. C\# nekoliko spominja na programski jezik Java, vendar se te dva jezika v kasnejših verzijah kar precej razlikujeta. Lep primer je implementacija genericov, ki je v C\# ustvarjena s pomočjo reifikacijo podatkov, v Javi pa s pomočjo posebne sintakse. 

\subsubsection{ANGLE}
% https://code.google.com/p/angleproject/
ANGLE: Almost Native Graphics Layer Engine
Angle je odprtokodni projekt, ki implementira OpenGL ES2.0 specifikacijo in jo strojno pospeši z Direct3D. Angle je v uporabi kot primarno zaledje za WebGL v brskalnikih Chrome in Firfox. Podpira DirectX9 do DirectX11.  


\subsection{Firefox OS}

Firefox OS je mobilni operacijski sistem, ki temelji na odprtih standardih spleta. Domorodne aplikacije pisane za Firefox OS so kar spletne aplikacije narejene po načelih HTML5, ki imajo posebne ovitke za klicanje sistemskih funkcij, kot je klicanje in dostop do senzorjev za lokacijo, pospeške in tako dalje.

Domorodne aplikacije gradimo z uporabo programskega jezika Javascript, za izgled in obliko aplikacij pa skrbita HTML5 in CSS. Aplikacije lahko z uporabo orodij v SDKju, preizkusimo tudi na mobilnih računalnikih.

Ker podpora temelji na standardih za spletne tehnologije ni podpore za OpenGL ES1.x, ki je na voljo na drugih mobilnih operacijskih sistemih. Je pa seveda podprt OpenGL ES 2.0 v obliki WebGLa.

Operacijski sistem Firefox OS sicer temelji na jedru Linux, ki služi kot platforma na katerem se nato zažene Gecko. Gecko je tako imenovani pogon za razporeditev, ki je prisoten v vseh verzijah brskalnika Firefox in pa tudi drugje. Ker Gecko deluje na različnih platformah, je Firefox OS možno naložiti tudi na druge naprave, tudi na RaspberryPi[\ref{sec:raspberryPi}]. 

\subsection{Ubuntu}

Operacijski sistem Ubuntu je najbolj popularen na Linuxu temelječi operacijski sistem na namiznih računalnikih. Podjetje, ki razvija operacijski sistem Ubuntu ima vizijo spraviti podobno funkcionalnost tudi na mobilne naprave - telefone in tablice. Prva naprava, ki bo izšla z mobilnim operacijskim sistemom Ubuntu, je telefon Edge in je bila napovedana 22. julija 2013. Pred tem so radovedni razvijalci lahko naložili operacijski sistem na mobilni telefon Nexus 4, ki ima privzeto naložen operacijski sistem Android.

% http://developer.ubuntu.com/resources/programming-languages/qml/
Canonical za razvoj aplikacij za mobilni operacijski sistem Ubuntu priporoča uporabo programskega jezika QML (Qt Meta Language). QML je programski jezik namenjen lažjemu razvoju uporabniškega vmesnika. 

Dokumentacija za Ubuntu Phone še ni popolna, ampak kot se da trenutno razbrati je uporaba OpenGLa podprta s strani qt 3d iniciative. Za pisanje grafično intenzivnih aplikacij se tako namesto QMLa priporoča C++.

O samih zmogljivostih telefona in tablic se sicer še ne ve veliko, pričakuje pa se, da bo podprt vsaj standard za OpenGL ES2.0 aplikacije. 

\subsection{Sailfish OS}

Sailfish je operacijski sistem temelječ na Linux jedru, namenjen mobilnim telefonom in drugim napravam. Podobno kot operacijski sistem Ubuntu bo tudi Sailfish uporabljal QML in Qt [\ref{sec:qt}] za razvoj domorodnih aplikacij.

Tudi kar se tiče podpore za razvoj grafično intenzivnih aplikacij je slično Ubuntu mobilnemu operacijskemu sistemu. Dostop do OpenGL programskega vmesnika je mogoč s pomočjo QTVIEWa in nam omogoča grajenje. Razlike med Ubuntu in Sailfish so minimalne, tako da bomo lahko teoretično aplikacijo pisali za oba sistema brez večjih sprememb.

\section{Računalniki na eni plošči}

\subsection{Raspberry Pi}
\label{sec:raspberryPi}

Raspberry PI je majhen računalnik, velikosti kreditne kartice, ki je bil razvit za promocijo učenja računalniške znanosti. Računalnik vsebuje 700 MHz ARM procesor, 256 (ali 512) MB delovnega pomnilnika in grafično procesno enoto VideoCore IV s 250MHz, ki podpira OpenGL ES 2.0.

Raspberry Pi je zanimiva mešanica med mobilnimi in namiznimi računalniki. Po svoji strojni opremi je sicer zelo podoben mobilnim napravam, vendar na njem ne teče mobilni operacijski sistem. Na Raspberry Piju je mogoče poganjati operacijske sisteme, ki so značilni za namizne računalnike. Najbolj pogosto uporabljena je malce prirejena verzija Linux distribucije Debian, uradno možno pa je naložiti tudi distribuciji Arch Linux in Fedora.

Čip za izrisovanje grafike na Raspberry Piju ima popolno podporo za OpenGL ES2.0 in je kljub omejitvam zmožen poganjati zahtevnejše aplikacije.  Lep primer je računalniška igra Minecraft.

% http://makezine.com/2013/04/15/arduino-uno-vs-beaglebone-vs-raspberry-pi/

\subsection{BeagleBone}
\label{sec:beagleBone}

Podobno kot Raspberry Pi je tudi BeagleBone majhen računalnik, ki je sposoben poganjati Linux sistem. BeagleBone ima malenkost hitrejši ARM procesor z 720MHz in 256 MB RAMa. Obstajajo tudi bolj zmogljive verzije, ki imajo do 1GHz procesor in 512MB RAMa (BeagleBone Black). Enota za grafično procesiranje PowerVR je zmožna poganjati tako 3D kot 2D OpenGL aplikacije.

\section{Skupne zmogljivosti}

Kot lahko vidimo iz navedenih primerov platform se ne moremo odločiti za en sam programski jezik in en sam programski vmesnik, s katerima bi pokrili vse možne platforme. Največji del trga lahko pokrijemo, če se odločimo za OpenGL ES 2.0 programski vmesnik z programskim jezikom C++, saj s tem lahko razvijamo grafično intenzivne aplikacije na Wndows, Linux in Mac OSX, kot tudi na mobilnih platformah Android, iOS... Kot bomo videli v nadaljevanju obstajajo tudi možnosti kako razširiti podporo.

Kot smo videli pri posameznih primerih ima večina mobilnih naprav podporo za OpenGL ES2.0, tako da je glavni problem pri razvoju medplatformnih aplikacijah premagati omejenost na programske jezike za določeno platformo. Večina mobilnih operacijskih sistemih ima svoj programski jezik, ki je v uporabi za pisanje domorodnih aplikacij. Načinov premagovanja teh ovir je več in nekaj si jih bomo tudi ogledali. Najbolj popularno je prevajati iz enega jezika v drugega na primer napišemo aplikacijo v programskem jeziku Java in potem s posebnimi orodji prevedemo v kodo, ki jo razumejo tudi C\# prevajalniki.

Na večini naprav je tudi na voljo tak ali drugačen spletni brskalnik. Spletne aplikacije, ki tečejo znotraj brskalnika, za izvajanje uporabljajo programski jezik Javascript, ki je tako eden redkih jezikov, ki je na voljo na vseh napravah. Spletne aplikacije so lahko tudi grafično intenzivne in za svoje delovanje lahko celo uporabljajo dostop do grafičnega procesorja.




