%%%%%%%%%%%%%%%%%%%%%%%%%%%%%%%%%%%%%%%%
% povzetek 
\addcontentsline{toc}{chapter}{Povzetek}
\chapter*{Povzetek}

V diplomskem delu smo naredili pregled metod za razvoj grafično intenzivnih aplikacij na različnih platformah. Raziskali smo možne platforme in si ogledali razlike med njimi. Poseben poudarek smo namenili mobilnim napravam. V nadaljevanju diplomskega dela smo si ogledali metode in orodja, s katerimi je možno premostiti razlike med platformami. Štiri metode smo opredelili bolj podrobno in za vsako od teh razvili testno aplikacijo. Raziskali smo tudi računanje na grafičnih procesnih enotah, saj lahko postane pomemben faktor pri razvoju medplatformnih aplikacij v prihodnosti. Ugotovili smo, da uporaba obravnavanih metod pohitri in poenostavi razvoj medplatformnih aplikacij.

\subsection*{Ključne besede:}

grafično intenzivne aplikacije, platforme, razvojna orodja, mobilne naprave, spletne tehnologije

% prazna stran
\clearemptydoublepage

%%%%%%%%%%%%%%%%%%%%%%%%%%%%%%%%%%%%%%%%
% abstract
\selectlanguage{english}
\addcontentsline{toc}{chapter}{Abstract}
\chapter*{Abstract}


We have investigated methods of developing cross-platform graphics-intense applications. We have researched different platforms and pointed out differences between them. Mobile platforms have received special attention. We have taken a look into methods and tools for overcoming differences between the platforms. We studied four methods more thoroughly and implemented test applications for each of those. Additionaly, we have researched computing on graphics processing units as this may become an important factor for developing cross-platform applications in the future. We concluded that cross-platform applications can be developed faster and easier by exerting the discussed methods.

\subsection*{Keywords:}

graphics-intensive applications, platforms, development tools, mobile devices, web technologies 

\selectlanguage{slovene}