\chapter{Uvod}
Datoteka {\tt diploma.tex} na kratko opisuje, kako se pisanja diplomskega dela lotimo z uporabo programskega pateka \LaTeX. V tem dokumentu bomo predstavili nekaj njegovih prednosti in hib. Kar se slednjih tiče, mi pride na misel ena sama. Ko se srečamo z njim nam izgleda kot kislo jabolko, nismo prepričani, da bi želeli vanj ugrizniti. Lahko pa z njim pripravimo odličen zavitek ali pa pridemo na okus.

Česa od tega dokumenta ne pričakujte? Izkušeni uporabniki \LaTeX{}a bi vse skupaj zastavili 
drugače. Morda bi napisali posebno razredno datoteko (\emph{class file}) --- v resnici priredili katero od obstoječih ---, v datoteki {\tt diploma.tex} ohranili samo najbolj grobo strukturo in vanjo vključevali  posamezna po\-glav\-ja. Hkrati s pisanjem teksta bi poskrbeli tudi za stvarno kazalo ({\tt makeindex}), literaturo pa bi citirali z uporabo {\BibTeX}{a}. Tega, skratka, v tem dokumentu ne boste našli.

Kaj vseeno najdemo. V Poglavju~\ref{ch1} bomo na hitro spoznali besedilne konstrukte kot so izreki, enačbe in dokazi. Naučili se bomo, kako se na njih sklicujemo. Poglavje~\ref{ch2} bo predstavilo vključevanje plovk: slik in tabel. V Poglavju~\ref{ch3} se bomo srečali s sklicevanjem na literaturo.
Sledil bo samo še zaključek.

\chapter{Sklicevanje na besedilne konstrukte}
\label{ch1}
Matematična ali popolna indukcija je eno prvih orodij, ki jih spoznamo za dokazovanje trditev pri matematičnih predmetih. 
\begin{izrek}
\label{iz:1}
Za vsako naravno število $n$ velja
\begin{equation}
n < 2^n.
\label{eq:1}
\end{equation}
\end{izrek}
\begin{dokaz}
Dokazovanje z indukcijo zahteva, da neenakost~\eqref{eq:1} najprej preverimo za najmanjše naravno število --- $0$. Res, ker je $0 < 1 = 2^0$, je neenačba~\eqref{eq:1} za $n=0$ izpolnjena.

Sledi indukcijski korak. S predpostavko, da je neenakost~\eqref{eq:1} veljavna pri nekem naravnem številu $n$, je potrebno pokazati, da je ista neenakost v veljavi tudi pri njegovem nasledniku --- naravnem številu $n+1$. Računajmo.
\begin{align}
n+1 &< 2^n + 1  \label{eq:2}\\
    &\le 2^n + 2^n \label{eq:3}\\
    &= 2^{n+1} \nonumber
\end{align} 
Neenakost~\eqref{eq:2} je posledica indukcijske predpostavke, neenakost~\eqref{eq:3} pa enostavno dejstvo, da je za vsako naravno število $n$ izraz $2^n$ vsaj tako velik kot 1. S tem je dokaz Izreka~\ref{iz:1} zaključen.
\end{dokaz}

Opazimo, da je \LaTeX\ številko izreka podredil številki poglavja.


\chapter{Plovke: slike in tabele}
\label{ch2}
Slike in daljše tabele praviloma vključujemo v dokument kot plovke. Pozicija plovke v končnem izdelku ni pogojena s tekom besedila, temveč z izgledom strani. \LaTeX\ bo skušal plovko postaviti samostojno, praviloma na vrh strani, na kateri se na takšno plovko prvič sklicujemo. Pri tem pa bo na vsako stran končnega izdelka želel postaviti tudi sorazmerno velik del besedila. V skrajnem primeru, če imamo res preveč plovk, se bo odločil za stran popolnoma zapolnjeno s plovkami.




\section{Formati slik}
Bitne slike, vektorske slike, kakršnekoli slike, z \LaTeX{}om lahko vključimo vse. 
Slika~\ref{pic1} je v {\tt .pdf} formatu.
\begin{figure}
\begin{center}
\includegraphics[width=10cm]{pic1.pdf}
\end{center}
\caption{Herschelov graf, vektorska grafika.}
\label{pic1}
\end{figure}
Pa res lahko vključimo slike katerihkoli formatov? Žal ne. Programski paket \LaTeX\ lahko uporabljamo v več dialektih. Ukaz {\tt latex} ne mara vključenih slik v formatu Portable Document Format {\tt .pdf}, ukaz {\tt pdflatex} pa ne prebavi slik v Encapsulated Postscript Formatu {\tt .eps}. 
Strnjeno v Tabeli~\ref{tbl:1}.

\begin{table}
\begin{center}
\begin{tabular}{l|ccc}
ukaz/format & {\tt .pdf} & {\tt .eps} & ostali formati \\ \hline
{\tt pdflatex} & da & ne & da \\
{\tt latex}   & ne & da  & da
\end{tabular}
\end{center}
\caption{}
\label{tbl:1}
\end{table}

Nasvet? Odločite se za uporabo ukaza {\tt pdflatex}. Vaš izdelek bo brez vmesnih stopenj na voljo v {.pdf} formatu in ga lahko odnesete v vsako tiskarno. Če morate na vsak način vključiti sliko, ki jo imate v {\tt .eps} formatu, jo vnaprej pretvorite v alternativni format, denimo {\tt .pdf}.

Včasih se da v okolju za uporabo programskega paketa \LaTeX\ nastaviti na kakšen način bomo prebavljali vhodne dokumente. Spustni meni na Sliki~\ref{pic2} odkriva uporabo \LaTeX{}a v njegovi pdf inkarnaciji --- {\tt pdflatex}.
\begin{figure}
\begin{center}
\includegraphics[width=10cm]{pic2.png}
\end{center}
\caption{Kateri dialekt uporabljati?}
\label{pic2}
\end{figure} 

Vključena Slika~\ref{pic2} je seveda bitna.

Kaj pa stran iz študentskega referata?\label{pp}
Tudi njo lahko vključimo v dokument. Toda ne kot plovko.

