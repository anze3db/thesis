\chapter{Uvod}

Mobilne naprave dandanes postajajo vse bolj vsakdanje. Pametni telefoni imajo v sebi več procesorske moči, kot namizni računalniki izpred parih let. Poleg procesorske moči praviloma vsebujejo tudi grafične procesne enote, ki jih razvijalci lahko izkoristijo za razvoj grafično intenzivnih aplikacij. Poleg telefonov pa so se pojavili tudi tablični račnalniki, ki imajo praviloma še boljše karakteristike kot pametni telefoni. 

Med posameznimi proizvajalci telefonov in tablic obstajajo velike razlike v razvojnem okolju. Vsak izmed mobilnih operacijskih sistemov uporablja drug programski jezik za razvoj nativnih aplikacij pa tudi pri izvedbi različnih knjižnic (npr. OpenGL ES) se porajajo razlike. Razvoj grafične aplikacije, ki bi jo napisali enkrat in bi delovala povsod, je tako skorajda nemogoč.

Problem postane še težji, če želimo poleg vseh mobilnih naprav podpreti še namizne računalnike. Sedaj imamo poleg različnih programskih jezikov in knjižnic, še različne možnosti interakcije z uporabnikom - vnos z dotikom na mobilnih napravah in vnos z miško in tipkovnico na namiznih računalnikih.
